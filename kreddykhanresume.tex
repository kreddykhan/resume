% https://github.com/plorcupine/latex-fontawesome/blob/master/fontawesome.sty
%!TEX TS-program = xelatex
%!TEX encoding = UTF-8 Unicode

\documentclass[10pt, a4paper]{article}
\usepackage{mathptmx}
\usepackage{anyfontsize}
\usepackage{t1enc}
\usepackage{fontawesome}
\usepackage{xcolor}

% LAYOUT
%--------------------------------
% Margins
\usepackage{geometry}
% Sets specific for the page. Note the left/right difference is to allow for the wide left margin notes
\geometry{letterpaper, left=50mm, right=20mm, top=13mm, bottom=13mm}
\usepackage{scrextend}

% Do not indent paragraphs
\setlength\parindent{0in}

% TYPOGRAPHY
%--------------------------------
\usepackage{fontspec}
\usepackage{xunicode}
\usepackage{xltxtra}
% converts LaTeX specials (quotes, dashes etc.) to Unicode
\defaultfontfeatures{Mapping=tex-text}
\setromanfont [Ligatures={Common}]{Minion Pro}
% Cool ampersand
\newcommand{\amper}{{\fontspec[Scale=.95]{Minion Pro}\selectfont\itshape\&}}

% Aligns along left axis (Makes the right side the ragged side)
\raggedright

% MARGIN NOTES
%--------------------------------
% Use the marginnote package to write marginnotes
\usepackage{marginnote}
% Set the command \note to create a marginnote
\newcommand{\note}[1]{\marginnote{\scriptsize #1}}
% Sets the marginnotes to be ragged left
\renewcommand*{\raggedleftmarginnote}{}
\pagenumbering{gobble}

% HEADINGS
%--------------------------------
\usepackage{sectsty}
\usepackage[normalem]{ulem}

\sectionfont{\rmfamily\mdseries\large\textbf}
\subsectionfont{\rmfamily\mdseries\scshape\normalsize}
\subsubsectionfont{\rmfamily\bfseries\upshape\normalsize}

% PDF SETUP
%--------------------------------
\usepackage{hyperref}
\hypersetup
{
  pdfauthor={Karishma Reddy Khan},
  pdfsubject={Karishma Reddy Khan's Resume},
  pdftitle={Karishma Reddy Khan's Resume},
  colorlinks, breaklinks, xetex, bookmarks,
  filecolor=black,
  urlcolor=[rgb]{0.117,0.682,1},
  linkcolor=[rgb]{0.117,0.682,0.858},
  citecolor=[rgb]{0.117,0.682,0.858}
}

% LIST ENVIRONMENT
%--------------------------------
\usepackage{paralist}
\setdefaultleftmargin{.5cm}{2cm}{}{}{}{}
\renewenvironment{itemize}[1]{\begin{compactitem}#1}{\end{compactitem}}

% MARGIN ENVIRONMENT
%--------------------------------
\newenvironment{setmargins}[1]
{
% Determines whether the margin is on the left or right
\reversemarginpar
% Sets the margin width
\marginparwidth=35mm
% Sets the separation between the margin note and the body of the text
\setlength{\marginparsep}{5mm}
% Creates a note with the specified size which holds the variable sent in as one. This is the section heading
\note{\fontsize{11}{12}\selectfont\textbf{\Large #1\\[1ex]}}
\ \\
% Creates the line underneath the section heading
% Note that a letterpaper has a paper width of 216mm
% We have a left margin of 50mm with 35mm margin note width and 5mm separation so a starting point of -40leaves a space of 10mm from the edge
% We want the same separation on the other end so we need a length of 216-20 = 196mm
\noindent\hspace{-40mm}\rule{196mm}{0.4pt}\\

}

% DOCUMENT
%--------------------------------
\begin{document}

\begin{addmargin}[-30mm]{0mm}
  \begin{center}
    {\LARGE Karishma Reddy Khan}\\[.2cm]
    % \faEnvelope \enskip \href{mailto:kreddykhan@brandeis.edu}{kreddykhan@brandeis.edu}\\
    % \faGithubAlt \enskip \href{https://github.com/kreddykhan}{kreddykhan}\\
    % \faLinkedinSquare \enskip \href{https://www.linkedin.com/pub/karishma-reddy-khan/a8/604/77b}{karishmareddykhan}\\
    \href{mailto:kreddykhan@brandeis.edu}{kreddykhan@brandeis.edu}\\
    \href{https://github.com/kreddykhan}{github.com/kreddykhan}\\
    \href{https://www.linkedin.com/pub/karishma-reddy-khan/a8/604/77b}{linkedin.com/in/karishmareddykhan}

  \end{center}
\end{addmargin}
\ \\[.3cm]

\begin{setmargins}{Education}
  \note{\textbf{\small Graduation: May 2021}}
  \textbf{Brandeis University | Waltham, MA}\\
  PhD | Computer Science\\[.2cm]

  \note{\textbf{\small Graduation: May 2017}}
  \textbf{Brandeis University | Waltham, MA}\\
  Master of Arts | Computer Science\\[.2cm]

  \note{\textbf{\small Graduation: May 2015}}
  \textbf{Mount Holyoke College | South Hadley, MA}\\
  Bachelor of Arts | Magna Cum Laude\\
  Majors: Physics and Theatre | Minor: Electrical Engineering
\end{setmargins}
\ \\

\begin{setmargins}{Experience}
  % \note{\textbf{\small Aug 2015--Dec 2016}}
  % \textbf{Brandeis University Computer Science Department | Waltham, MA} \\
  % \emph{Head Teaching Assistant: Discrete Structures, Scientific Data Processing, Software Engineering Scalability}
  % \ \\[.3cm]

  \note{\textbf{\small June 2016--Aug 2016}}
  \textbf{High Energy Physics Lab, Brandeis University Physics Department | Waltham, MA} \\
  \emph{Programmer}
  \begin{itemize}
    \item Developed a Matlab GUI to simulate experiments to map the human eye
    \item Developed image stitching algorithms to stitch together experimental data results
  \end{itemize}
  \ \\[.3cm]

  \note{\textbf{\small Sep 2015--Dec 2015}}
  \textbf{SAXSLAB U.S.A. | Northampton, MA} \\
  \emph{Developer}
  \begin{itemize}
    \item Company manufactures X-Ray scattering devices and analyzes scattering data
    \item Updated pre-existing Matlab 2012a GUI code to be compatible with Matlab 2015a
  \end{itemize}
  \ \\[.3cm]

  \note{\textbf{\small June 2015--Aug 2015}}
  \textbf{Molmex Scientific | Northampton, MA} \\
  \emph{Intern}
  \begin{itemize}
    \item Company designed and manufactured Small Angle X-Ray scattering devices
    \item Designed 3D models in SolidWorks which are currently in use on the devices
    \item Improved user interface of scattering devices using \emph{spec}, a C-like language
  \end{itemize}
  \ \\[.3cm]

  \note{\textbf{\small May 2012--May 2015}}
  \textbf{Mount Holyoke College, Atomic Force Microscopy Lab | South Hadley, MA} \\
  \emph{Research Fellow with Dr. Katherine Aidala}
  \begin{itemize}
    \item Researched solar cell applications of nanoscale semi-conductors called Quantum Dots
    \item Studied crack formation in sub-monolayers of PbS Quantum Dots
  \end{itemize}
  \ \\[.3cm]

  \note{\textbf{\small June 2013--Aug 2013}}
  \textbf{Fermi National Accelerator Lab | Batavia, IL} \\
  \emph{Research Student}
  \begin{itemize}
    \item Worked with Wire Position Monitors (WPMs) used to detect motion in Linear Accelerator Cavities
    \item Developed a Matlab GUI to analyze data from WPMs that is still in use
    \item Demonstrated that Matlab is compatible with Fermilab's accelerator network
  \end{itemize}
\end{setmargins}
\ \\

\begin{setmargins}{Projects}
  \note{\textbf{\small Oct 2016 -- Ongoing}}
  \note{\raggedleft\href{https://sites.google.com/a/brandeis.edu/ma43phd/}{\textcolor{black}{\normalsize\faGlobe}} \href{https://github.com/kreddykhan/quantum-escapement}{\textcolor{black}{\normalsize\faGithubAlt}}}
  \textbf{Quantum Escapement:} Escape the room style game built using Blender and Python\\
  \note{\textbf{\small Sep 2016 -- Ongoing}}
  \note{\raggedleft\href{https://github.com/kreddykhan/nanotwitter}{\textcolor{black}{\normalsize\faGithubAlt}}}
  \textbf{NanoTwitter:} Small scale Twitter app built using Ruby and Sinatra as a study in scalability\\
  \note{\textbf{\small Aug 2016 -- Ongoing}}
  \note{\raggedleft\href{https://github.com/kreddykhan/CCD-Camera-Simulator}{\textcolor{black}{\normalsize\faGithubAlt}}}
  \textbf{CCD:} Matlab program that simulates a CCD camera using pixel bining and Riemann sums\\
  \note{\textbf{\small Jan 2015 -- May 2015}}
  \note{\raggedleft\href{https://sites.google.com/a/mtholyoke.edu/cs-243-spring-15---reddy22k/home/project-2-remote-control-car}{\textcolor{black}{\normalsize\faGlobe}} \href{https://github.com/kreddykhan/turtle}{\textcolor{black}{\normalsize\faGithubAlt}}}
  \textbf{Turtle 2.0:} Arduino robot with IR driven object avoidance and RF dynamic communication\\
  % \note{\textbf{\small Jan 2015 -- May 2015}}
  % \textbf{GPS:} Robotic car capable of receiving and navigating towards GPS co-ordinates\\
\end{setmargins}
\ \\

\begin{setmargins}{Skills}
  \note{\textbf{\small Software}}
  \textbf{Languages:} Java, Matlab, Ruby, Scheme, Python, JavaScript, HTML, spec\\
  \textbf{Frameworks:} Sinatra, Ruby on Rails\\
  \textbf{3D Animation:} SolidWorks, Blender\\
  \textbf{Tooling:} Git, \LaTeX, MySQL
  \ \\[.3cm]
  \note{\textbf{\small Hardware}}
  \textbf{Electronics:} Arduino, analog and digital circuitry, oscilloscopes, soldering\\
  \textbf{Lab Skills:} Atomic Force Microscopy, spin coating, plasma cleaning, machining
\end{setmargins}
% \ \\

% \begin{setmargins}{Coursework}
%  \textbf{Computer Science:} Scalability, 3D Animation, Operating Systems, Software Engineering Capstone, Functional Programming, Artificial Intelligence, Data Structures, Java Programming, Robotics \\
%  \textbf{Physics:} Nanoelectronics, Circuit Analysis, Electronics, Statistical Mechanics, Quantum Mechanics, Electromagnetism
% \end{setmargins}
% \ \\

% \begin{setmargins}{Interests}
%   Theatre, particularly stage management, writing and directing, comic books, video games, travel\\
% \end{setmargins}

\end{document}
